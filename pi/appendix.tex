\begin{appendixenv}[Apêndices]
    \section{Valor de Pi} \label{apx:pi-number}

    Sendo, $\pi_n$, $n$ a quantidade de casas decimais, temos:

    \input{generated/pi_takano.tex}

    \cleardoublepage
    \section{Código fonte e Dados} \label{apx:source-code}

    \subsection{Convergência para o método de \textit{Monte Carlo}}
    \label{sec:convergence_coding_at_monte_carlo}
    
    \lstinputlisting[style=macrocode,caption={Código fonte para a convergência para o método de \textit{Monte Carlo}},name=convergence_coding_at_monte_carlo,language=R]{script_convergence.R}
    
    \newpage
    
    \subsection{Histograma dos valores obtidos no método de Monte Carlo}
    \label{sec:histogram_coding_at_monte_carlo}
    
    \lstinputlisting[style=macrocode,caption={Código fonte para o histograma dos valores obtidos no método de Monte Carlo},name=histogram_coding_at_monte_carlo,language=R]{script_histogram.R}
    
    \newpage
    
    \end{appendixenv}