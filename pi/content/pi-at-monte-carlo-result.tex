\begin{graphic}[H]
   \centering
   \caption{Convergência para o método de \textit{Monte Carlo}}
   \includegraphics[width=0.95\textwidth,keepaspectratio]{generated/plot_monteCarlo}
   Fonte: Elaborado pelo próprio autor
   \label{fig:convergence_monteCalor}
\end{graphic}%

Observando que os valores coletados na \autoref{fig:convergence_monteCalor} convergem com flutuações nos valores, pois os valores são obtidos de forma aleatória. Então, podemos interpretar estatisticamente este resultado realizando uma hipótese de normalidade na amostra.

\begin{figure}[H]
   \centering
   \caption{Histograma dos valores obtidos no método de \textit{Monte Carlo}, com curva de densidade, aplicado à coleta de $\pi$}
   \includegraphics[width=0.95\columnwidth]{generated/histogram_monteCarlo}
   Fonte: Elaborado pelo próprio autor
   \label{fig:histogram_monteCalor}
\end{figure}%

Devido às propriedades da função, temos:
\begin{itemize}
   \item O ponto $x = \mu$ é um ponto de máximo absoluto, sendo o único ponto crítico do gráfico, representado pela linha pontilhada central da \autoref{fig:histogram_monteCalor}.
   \item Possui dois pontos de inflexão $x = \mu - \sigma$ e $x = \mu - \sigma$, representado pelas linhas pontilhadas nas extremidades do gráfico.
   \item A área entre $\mu \pm \sigma$ representa $68\%$ da área total do gráfico, ou seja, $68\%$ dos valores de $\pi$ estão presentes nesse intervalo.
\end{itemize}
Defina:
$mean(var)$, média aritmética dos valores de $\pi$ obtidos.
\begin{gather*}
   var = \{x_{1},\ x_{2},\ \dots,\ x_{n}\} \\
   \mu = \bar{var} = \frac{1}{n} \sum_{i=1}^{n} x_{i} = \frac{x_{1} + x_{2} + \dots + x_{n}}{n}
\end{gather*}

$sd(var)$, desvio padrão dos valores obtidos, ou seja, medida de dispersão com as mesmas unidades que a amostra.
\begin{gather*}
   \sigma = sd = \sqrt{\frac{1}{n-1}\sum_{i=1}^{n}\left(x_{i}-\bar{var}\right)^{2}}
\end{gather*}

$sd(var) / sqrt(m)$, desvio padrão da média ($\bar{var}$), no limite para $m$ grande. O desvio padrão da média ($\xi$) caracteriza as flutuações dos valores $\bar{var}_{i}$ em torna da média população $\mu$.
\begin{gather*}
   \xi = \frac{\sigma}{\sqrt{m}}
\end{gather*}

E realizando testes estatísticos nos dados para determinar se os dados afastam ou não da distribuição normal.

\begin{lstlisting}[name=coding_at_monte_carlo,language=R,style=macrocode,numbers=none]
shapiro.test(var)
\end{lstlisting}
\begin{lstlisting}[name=coding_at_monte_carlo,style=macrocode,frame=single,numbers=none]
[1]	Shapiro-Wilk normality test

data:  var
W = 0.9987, p-value = 0.6886
\end{lstlisting}

Observando que o \verb|valor-p| retornado pelo teste é maior que $0{,}05$, fortalecendo a hipótese que os dados são normalmente distribuídos.

Além disso, tem-se:
\begin{lstlisting}[name=coding_at_monte_carlo,language=R,style=macrocode,numbers=none]
library(nortest)
ad.test(var)
\end{lstlisting}
\begin{lstlisting}[name=coding_at_monte_carlo,style=macrocode,frame=single,numbers=none]
[1]   Anderson-Darling normality test

data:  var
A = 0.2726, p-value = 0.6684
\end{lstlisting}

\begin{lstlisting}[name=coding_at_monte_carlo,language=R,style=macrocode,numbers=none]
ks.test(var, "pnorm", mean(var), sd(var))
\end{lstlisting}
\begin{lstlisting}[name=coding_at_monte_carlo,style=macrocode,frame=single,numbers=none]
[1]	Asymptotic one-sample Kolmogorov-Smirnov test

data:  var
D = 0.015002, p-value = 0.978
alternative hypothesis: two-sided
\end{lstlisting}

O \verb|valor-p| em todos os testes anteriores foi superior a $0{,}05$, fortalecendo a nossa hipótese inicial de que a normalidade não é rejeitada, como demonstrado na \autoref{fig:histogram_monteCalor}. Então, apresentando erros de medição aleatórios.